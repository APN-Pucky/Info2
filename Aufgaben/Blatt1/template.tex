%%%%%%%%%%%%%%%%%%%%%%%%%%%%%%%%%%%%%%%%%%%%%%%%%%%%%%%%%%%%%%%%%%%%%%%%%%%%%%%%%%%%%%%%%
%             template.tex                                                              %
%                                                                                       %
%            Author: Sergej Lewin 10/2008                                               %
%                                                                                       %    
% !!!Man braucht noch die Datei Ueb.sty (im gleichen Ordner wie die Hauptdatei)!!!      %
%%%%%%%%%%%%%%%%%%%%%%%%%%%%%%%%%%%%%%%%%%%%%%%%%%%%%%%%%%%%%%%%%%%%%%%%%%%%%%%%%%%%%%%%%
\documentclass[a4paper,11pt]{article}             % bestimmt das Aussehen eines Dokuments
\usepackage{Ueb}                                  % vordefinierte Makros
\usepackage{enumitem}
\renewcommand{\labelenumi}{(\alph{enumi})}

%!!!!anpassen an das Betriebssystem!!!, um Umlaute zu verwenden
\usepackage[utf8]{inputenc}                      %Linux
%\usepackage[latin1]{inputenc}                    %Windows
%\usepackage[applemac]{inputenc}                  %Mac



%Namen und Matrikelnummern anpassen
%\zweinamen{Name1}{Matrikelnummer1}{Name2}{Matrikelnummer2} %2er Gruppen
\dreinamen{Alexander Neuwirth}{439218}{Leonhard Segger}{440145}{Jonathan Sigrist}{441760} %3er Gruppe

%Briefkastennummer anpassen. z. B. \briefkasten{104}
\briefkasten{}

%Termin der Uebungsgruppe und Raum anpassen z. B. \termin{Mo. 12-14 , SR2}
\termin{Fr. 08-10, SR217}

%Blattnummer anpassen z. B. \blatt{5}
\blatt{1}

\begin{document}

\Aufgabe{1}

Mit zwei Variablen gibt es vier unterschiedliche Eingabekombinationen.
Dadurch hat jede Funktion genau vier mögliche Eingaben und dementsprechend vier zugeordnete Ergebnisse.
Da sich jede Funktion von allen anderen Unterscheiden soll und nur mit Bits, also zwei Zuständen arbeitet gibt es genau $2^4 = 16$ verschiedene Funktionen der Form $f (x, y) : \{0,1\} \times \{0, 1\} \rightarrow \{0, 1\}$.

\Aufgabe{2}

NAND:\\*
NOT A = A NAND A\\*
A AND B = NOT (A NAND B) = (A NAND B) NAND (A NAND B)\\

NOR:\\*
NOT A = A NOR A\\*
A AND B = NOT A NOR NOT B = (A NOR A) NOR (B NOR B)\\

Aus den Funktionen NOT und AND lassen sich nun alle weiteren Funktionen durch Kombination erzeugen.

\Aufgabe{3}

Teilt man eine beliebige Gruppe mit n Elementen möglichst klein auf, so erhällt man stets n Untergruppen mit jeweils einem Element.
Der Wert des Elements in einer dieser Gruppen entspricht dann dem Gesammtwert dieser Gruppe. 
Durch Addition der Werte zweier Gruppen erhält man eine einzige Gruppe mit dem Wert beider Untergruppen.
So können nur n-1 Additionen durchgeführt werden, da dann nur noch ein Element übrig bleibt.\\*

Wenn man nun bei dem logarithmischen Bitsummen-Algorithmus jeweils die Gruppen so teilt, dass eine Untergruppe nur ein Element
und die andere Untergruppe die restlichen Elemente enthält, wird man am Ende auf genau n Ebenen kommen,
welche nun wie oben erklärt durch n-1 Additionen vereinigt werden müssen. Da das der Worst-Case ist, kann es somit nicht mehr als n-1 Additionen geben.

\Aufgabe{4}

\begin{enumerate}

\item Hallo

\item Welt
\item !

\end{enumerate}

\end{document}

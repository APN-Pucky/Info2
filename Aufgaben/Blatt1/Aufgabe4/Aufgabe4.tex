%%%%%%%%%%%%%%%%%%%%%%%%%%%%%%%%%%%%%%%%%%%%%%%%%%%%%%%%%%%%%%%%%%%%%%%%%%%%%%%%%%%%%%%%%
%             template.tex                                                              %
%                                                                                       %
%            Author: Sergej Lewin 10/2008                                               %
%                                                                                       %    
% !!!Man braucht noch die Datei Ueb.sty (im gleichen Ordner wie die Hauptdatei)!!!      %
%%%%%%%%%%%%%%%%%%%%%%%%%%%%%%%%%%%%%%%%%%%%%%%%%%%%%%%%%%%%%%%%%%%%%%%%%%%%%%%%%%%%%%%%%
\documentclass[a4paper,11pt]{article}             % bestimmt das Aussehen eines Dokuments
\usepackage{Ueb}                                  % vordefinierte Makros
\usepackage{enumitem}
\renewcommand{\labelenumi}{(\alph{enumi})}

%!!!!anpassen an das Betriebssystem!!!, um Umlaute zu verwenden
\usepackage[utf8]{inputenc}                      %Linux
%\usepackage[latin1]{inputenc}                    %Windows
%\usepackage[applemac]{inputenc}                  %Mac



%Namen und Matrikelnummern anpassen
%\zweinamen{Name1}{Matrikelnummer1}{Name2}{Matrikelnummer2} %2er Gruppen
\dreinamen{Alexander Neuwirth}{439218}{Leonhard Segger}{440145}{Jonathan Sigrist}{441760} %3er Gruppe

%Briefkastennummer anpassen. z. B. \briefkasten{104}
\briefkasten{}

%Termin der Uebungsgruppe und Raum anpassen z. B. \termin{Mo. 12-14 , SR2}
\termin{Fr. 08-10, SR217}

%Blattnummer anpassen z. B. \blatt{5}
\blatt{1}

\begin{document}

\Aufgabe{4}

\begin{enumerate}

\item Man betrachte die Multimengendarstellungen von $a$ und $b$ gegeben durch $a_m = PD_m(a)$ sowie $b_m = PD_m(b)$.\\
Dann ist $gcd_m = PD_m(gcd(a, b)) = a_m \cap b_m$.\\
Ebenso seien $a_r = a_m \setminus gcd_m$ und $b_r = b_m \setminus gcd_m$, sodass $a_m = a_r \cup gcd_m$ und $b_m = b_r \cup gcd_m$.\\
Daraus folgt, dass $a_m \cup b_m = a_r \cup b_r \cup gcd_m \cup gcd_m$.\\
Ebenso folgt direkt, dass $PD_m(\frac {a \cdot b}{gcd(a, b)}) = a_r \cup b_r \cup gcd_m = a_r \cup b_m = b_r \cup a_m$.\\
Da $PD_m(lcm(a, b)) = a_m \cup a_s = b_m \cup b_s$ und, da es sich um das kleinste gemeinsame Vielfache handelt, $a_s \cap b_s = \varnothing$ gilt, muss $a_r = a_s$ und $b_r = b_s$ sein.\\
Somit resultiert, dass $PD_m(\frac {a \cdot b}{gcd(a, b)}) = PD_m(lcm(a, b))$ bzw. $\frac {a \cdot b}{gcd(a, b)} = lcm(a, b)$.

\item Nach der ersten Teilaufgabe kann man das Problem mit $lcm(a, b) = \frac {a \cdot b}{gcd(a, b)}$ auf die Berechnung des größten gemeinsamen Teilers reduzieren. Dafür haben wir in der Vorlesung bereits den Euklidschen Algorithmus kennengelernt. \\
(siehe Programmieraufgaben innerhalb des .zip Archives)

\end{enumerate}

\end{document}

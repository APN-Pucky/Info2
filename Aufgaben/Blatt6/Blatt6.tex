%%%%%%%%%%%%%%%%%%%%%%%%%%%%%%%%%%%%%%%%%%%%%%%%%%%%%%%%%%%%%%%%%%%%%%%%%%%%%%%%%%%%%%%%%
%             template.tex                                                              %
%                                                                                       %
%            Author: Sergej Lewin 10/2008                                               %
%                                                                                       %    
% !!!Man braucht noch die Datei Ueb.sty (im gleichen Ordner wie die Hauptdatei)!!!      %
%%%%%%%%%%%%%%%%%%%%%%%%%%%%%%%%%%%%%%%%%%%%%%%%%%%%%%%%%%%%%%%%%%%%%%%%%%%%%%%%%%%%%%%%%
\documentclass[a4paper,11pt]{article}             % bestimmt das Aussehen eines Dokuments
\usepackage{Ueb}                                  % vordefinierte Makros
\usepackage{enumitem}
\renewcommand{\labelenumi}{(\alph{enumi})}
\renewcommand{\labelenumii}{(\roman{enumii})}

%!!!!anpassen an das Betriebssystem!!!, um Umlaute zu verwenden
\usepackage[utf8]{inputenc}                      %Linux
%\usepackage[latin1]{inputenc}                    %Windows
%\usepackage[applemac]{inputenc}                  %Mac



%Namen und Matrikelnummern anpassen
%\zweinamen{Name1}{Matrikelnummer1}{Name2}{Matrikelnummer2} %2er Gruppen
\dreinamen{Alexander Neuwirth}{439218}{Leonhard Segger}{440145}{Jonathan Sigrist}{441760} %3er Gruppe

%Briefkastennummer anpassen. z. B. \briefkasten{104}
\briefkasten{}

%Termin der Uebungsgruppe und Raum anpassen z. B. \termin{Mo. 12-14 , SR2}
\termin{Fr. 08-10, SR217}

%Blattnummer anpassen z. B. \blatt{5}
\blatt{6}

\begin{document}

\Aufgabe{20}

\begin{enumerate}
\item Der Algorithmus durchläuft das noch nicht sortierte Array und tauscht jeweils benachbarte Paare, sodass $\mathtt{a[i+1] >= a[i]}$. Dieser Vorgang wird so lange wiederholt, bis keine Arrayeinträge mehr vertauscht worden sind und somit alle Einträge in aufsteigender Folge sortiert sind.
\item Da die $\mathtt{run}$-Variable pro Durchlauf inkrementiert wird, erreicht sie irgendwann den Wert $\mathtt{a.length}$. Da nun die innere Schleife nicht mehr durchlaufen wird, bleibt auch die $\mathtt{vertauscht}$-Variable zwangsweise $\mathtt{false}$ und der Algorithmus terminiert.
\item Der worst-case ist $\mathcal{O}(n^2)$, da die äußere Schleife aufgrund der in (b) genannten Inkrementierung höchstens n-mal und die innere Schleife ebenso maximal n-mal durchlaufen wird. Dabei sei anzumerken, dass die innere Schleife pro Durchlauf der äußeren Schleife einmal weniger durchlaufen wird und somit die eigentliche Laufzeit $\frac{1}{2}n^2$ lautet, was in der $\mathcal{O}$-Notation auf die selbe Menge deutet.\\
Der Algorithmus unternimmt maximal $\frac{n\cdot(n-1)}{2}$-Tauschoperationen.

\end{enumerate}

\end{document}


%%%%%%%%%%%%%%%%%%%%%%%%%%%%%%%%%%%%%%%%%%%%%%%%%%%%%%%%%%%%%%%%%%%%%%%%%%%%%%%%%%%%%%%%%
%             template.tex                                                              %
%                                                                                       %
%            Author: Sergej Lewin 10/2008                                               %
%                                                                                       %    
% !!!Man braucht noch die Datei Ueb.sty (im gleichen Ordner wie die Hauptdatei)!!!      %
%%%%%%%%%%%%%%%%%%%%%%%%%%%%%%%%%%%%%%%%%%%%%%%%%%%%%%%%%%%%%%%%%%%%%%%%%%%%%%%%%%%%%%%%%
\documentclass[a4paper,11pt]{article}             % bestimmt das Aussehen eines Dokuments
\usepackage{Ueb}                                  % vordefinierte Makros
\usepackage{enumitem}
\renewcommand{\labelenumi}{(\alph{enumi})}
\renewcommand{\labelenumii}{(\roman{enumii})}

%!!!!anpassen an das Betriebssystem!!!, um Umlaute zu verwenden
\usepackage[utf8]{inputenc}                      %Linux
%\usepackage[latin1]{inputenc}                    %Windows
%\usepackage[applemac]{inputenc}                  %Mac



%Namen und Matrikelnummern anpassen
%\zweinamen{Name1}{Matrikelnummer1}{Name2}{Matrikelnummer2} %2er Gruppen
\dreinamen{Alexander Neuwirth}{439218}{Leonhard Segger}{440145}{Jonathan Sigrist}{441760} %3er Gruppe

%Briefkastennummer anpassen. z. B. \briefkasten{104}
\briefkasten{}

%Termin der Uebungsgruppe und Raum anpassen z. B. \termin{Mo. 12-14 , SR2}
\termin{Fr. 08-10, SR217}

%Blattnummer anpassen z. B. \blatt{5}
\blatt{8}

\begin{document}

\Aufgabe{26}
\begin{enumerate}
\item Da stets $M_{ij} = - M_{ji}$, gilt auf der Hauptdiagonalen ebenso $M_{ii} = -M_{ii} \Rightarrow 0 = -0$. Dementsprechend muss auf der Hauptdiagonalen $M_{ii} = 0$ sein.

\item Man muss nur die Werte für $j > i$ speichern, da die Hauptdiagonale immer $0$ ist und die andere Seite dieser durch die Forderung $M_{ij} = -M_{ji}$ bekannt sind.
\end{enumerate}

\Aufgabe{27}
\begin{enumerate}
\item Es seien die beiden Funktionen \texttt{min} und \texttt{max} definiert mit:
\begin{itemize}
\item
$\texttt{min: DICT -> ELEM}$\\
$\texttt{min(create) = ERROR}$\\
$\texttt{min(insert(E, create)) = E}$\\
$\texttt{min(insert(E, D)) = min}_{\texttt{ELEM}}\texttt{(E, min(D))}$

\item
$\texttt{max: DICT -> ELEM}$\\
$\texttt{max(create) = ERROR}$\\
$\texttt{max(insert(E, create)) = E}$\\
$\texttt{max(insert(E, D)) = max}_{\texttt{ELEM}}\texttt{(E, max(D))}$
\end{itemize}

\item Es seien die beiden Funktionen \texttt{succ} und \texttt{pred} definiert mit:
\begin{itemize}
\item
$\texttt{succ: ELEM x DICT -> ELEM}$\\
$\texttt{succ(E, insert(E, create)) = largestElem}$\\
$\texttt{succ(E, D) = if isequal}_{\texttt{ELEM}}\texttt{(E,min(D)) then min(delete(E,D)) else succ(E,delete(min(D),D))}$\\

\item
$\texttt{pred: ELEM x DICT -> ELEM}$\\
$\texttt{pred(E, insert(E, create)) = smallestElem}$\\
$\texttt{pred(E, D) = if isequal}_{\texttt{ELEM}}\texttt{(E,max(D)) then max(delete(E,D)) else pred(E,delete(max(D),D))}$\\

\end{itemize}
\end{enumerate}

\Aufgabe{28}
Die Laufzeit des Programms ist $n(2b+1)\cdot(2b+1) = \mathcal{O}(nb^2)$.

\Aufgabe{29}
\begin{enumerate}
\item Zuerst ist das Array mit $\mathcal O(n\log n)$ zu sortieren. Dann sucht man für jedes Element binär nach einem Partner-Element ($\mathcal O(n\log n)$) mit $\texttt{values[i]+values[j]} = \texttt{sum}$. Da alle Werte in $\texttt{values}$ paarweise verschieden sind, enthält das Ergebnis auch keine Duplikate.
\end{enumerate}

\Aufgabe{30}
\begin{enumerate}
\item
\begin{itemize}
\item $f_1=n^4$
\item $f_2=n\cdot\log^2n$
\item $f_3=n^2$
\item $f_4=\log^2n$
\end{itemize}

\item Die Funktionen $f(n) = \log_2(n^n) = n\cdot\log_2 n$ und $g(n)=n^2\log_2 n$ werden im Grenzfall $\lim\limits_{n\to\infty}\frac{f}{g}(n)=\lim\limits_{n\to\infty}\frac{1}{n}=0\Rightarrow f=\mathcal O(g)$. Das wurde bereits in Aufgabe 14 b) gezeigt.

\item $f=n\log_4 n = n\frac{\log_2 n}{\log_2 4} = \frac{1}{2} n\log_2 n$ und $g=n\log_2 n$. Da $f=c\cdot g$ mit $c=\frac{1}{2}$ gilt $f\in\Omega(g)$.

\item Sei $g=\log_2 n$. Dann $\exists c>0,n_0 \forall n>n_0 | f(n) \leq c\cdot g(n)$ mit $n_0 = 42$ und $c=1$, also $\forall n>42|f(n) = g(n)$ und somit $f\in\mathcal O(g)$.
\end{enumerate}

\end{document}


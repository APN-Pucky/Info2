%%%%%%%%%%%%%%%%%%%%%%%%%%%%%%%%%%%%%%%%%%%%%%%%%%%%%%%%%%%%%%%%%%%%%%%%%%%%%%%%%%%%%%%%%
%             template.tex                                                              %
%                                                                                       %
%            Author: Sergej Lewin 10/2008                                               %
%                                                                                       %    
% !!!Man braucht noch die Datei Ueb.sty (im gleichen Ordner wie die Hauptdatei)!!!      %
%%%%%%%%%%%%%%%%%%%%%%%%%%%%%%%%%%%%%%%%%%%%%%%%%%%%%%%%%%%%%%%%%%%%%%%%%%%%%%%%%%%%%%%%%
\documentclass[a4paper,11pt]{article}             % bestimmt das Aussehen eines Dokuments
\usepackage{Ueb}                                  % vordefinierte Makros
\usepackage{enumitem}
\renewcommand{\labelenumi}{(\alph{enumi})}

%!!!!anpassen an das Betriebssystem!!!, um Umlaute zu verwenden
\usepackage[utf8]{inputenc}                      %Linux
%\usepackage[latin1]{inputenc}                    %Windows
%\usepackage[applemac]{inputenc}                  %Mac



%Namen und Matrikelnummern anpassen
%\zweinamen{Name1}{Matrikelnummer1}{Name2}{Matrikelnummer2} %2er Gruppen
\dreinamen{Alexander Neuwirth}{439218}{Leonhard Segger}{440145}{Jonathan Sigrist}{441760} %3er Gruppe

%Briefkastennummer anpassen. z. B. \briefkasten{104}
\briefkasten{}

%Termin der Uebungsgruppe und Raum anpassen z. B. \termin{Mo. 12-14 , SR2}
\termin{Fr. 08-10, SR217}

%Blattnummer anpassen z. B. \blatt{5}
\blatt{2}

\begin{document}

\Aufgabe{5}

\begin{enumerate}

\item Die einzelnen Komponenten der Darstellung tragen zu der gesuchten Zahl der insgesammten Zahlendarstellungen bei.\\
Das Vorzeichen trägt mit einem Faktor von $2$ dazu bei.
Die 3 Oktalzahlen der Mantisse tragen genau $8^3 = 512$ dazu bei
und der Exponent kann genau $8^1 = 8$ Zahlen darstellen.\\
Zusammen ergibt das $2 \cdot 8^3 \cdot 8 = 8192$ verschiedene Zahlendarstellungen.

\item Da verschiedene Zahlendarstellungen dieselben Zahlen darstellen, fallen nun einige Zahlendarstellungen raus.\\
Da der Exponent auf einer Oktalbasis arbeitet, wird bei Inkrementierung des Exponenten um eins die Mantisse um eins nach rechts verschoben, also $d_1 = 0; d_2 = d_1; d_3 = d_2$ um dieselbe Zahl darzusellen.\\
Um eine Ordnung zu erhalten, soll der Exponent also möglichst gering gewählt werden, sodass $d_1$ nicht 0 wird. Dies ist mit Ausnahme von $e=0$.\\
Das Vorzeichen ist nur bei $d_1 = d_2 = d_3 = e = 0$ nicht relevant.
\begin{description}
\item [$e \neq 0$]
Man kann annehmen, dass $d_1 \neq 0$ ist. Somit gibt es $2 \cdot 7 \cdot 8^2 \cdot 7 = 6272$ verschiedene darstellbare Zahlen.
\item [$e = 0$]
Nun muss man berücksichtigen, dass $d_1$ auch kleiner als 1 werden kann. Somit ergeben sich weitere $2 \cdot 8^3 \cdot 1 = 1024$ verschiedene darstellbare Zahlen
\end{description}
Zählt man nun beide Teile zusammen und beachtet die 0 erhält man $7295$ verschiedene darstellbare Zahlen.

\end{enumerate}

\end{document}

%%%%%%%%%%%%%%%%%%%%%%%%%%%%%%%%%%%%%%%%%%%%%%%%%%%%%%%%%%%%%%%%%%%%%%%%%%%%%%%%%%%%%%%%%
%             template.tex                                                              %
%                                                                                       %
%            Author: Sergej Lewin 10/2008                                               %
%                                                                                       %    
% !!!Man braucht noch die Datei Ueb.sty (im gleichen Ordner wie die Hauptdatei)!!!      %
%%%%%%%%%%%%%%%%%%%%%%%%%%%%%%%%%%%%%%%%%%%%%%%%%%%%%%%%%%%%%%%%%%%%%%%%%%%%%%%%%%%%%%%%%
\documentclass[a4paper,11pt]{article}             % bestimmt das Aussehen eines Dokuments
\usepackage{Ueb}                                  % vordefinierte Makros
\usepackage{enumitem}
\renewcommand{\labelenumi}{(\alph{enumi})}

%!!!!anpassen an das Betriebssystem!!!, um Umlaute zu verwenden
\usepackage[utf8]{inputenc}                      %Linux
%\usepackage[latin1]{inputenc}                    %Windows
%\usepackage[applemac]{inputenc}                  %Mac



%Namen und Matrikelnummern anpassen
%\zweinamen{Name1}{Matrikelnummer1}{Name2}{Matrikelnummer2} %2er Gruppen
\dreinamen{Alexander Neuwirth}{439218}{Leonhard Segger}{440145}{Jonathan Sigrist}{441760} %3er Gruppe

%Briefkastennummer anpassen. z. B. \briefkasten{104}
\briefkasten{}

%Termin der Uebungsgruppe und Raum anpassen z. B. \termin{Mo. 12-14 , SR2}
\termin{Fr. 08-10, SR217}

%Blattnummer anpassen z. B. \blatt{5}
\blatt{2}

\begin{document}

\Aufgabe{7}

\begin{enumerate}

\item Da die Rechenzeit exponentiell ansteigt, sind bereits Fibunaccizahlen von 40 nicht mehr gut zu berechnen.
\item Hierbei ist die Rechenzeit linear, jedoch wird schnell ein Overflow produziert und die Werte stimmen auch nicht mehr.
\item 
\begin{enumerate}
\item 
modulares Zahlensystem:\\
$m_1 = 999$\\
$m_2 = 1000$\\
$m_3 = 1001$\\
\\
$\Rightarrow m = m_1 \cdot m_2 \cdot m_3$\\
\\
Zahlenbereich: $0 \leq r < 999999000$

\item
\begin{description}
\item [Formal]
$r_1 = r mod 999$\\
$r_2 = r mod 1000$\\
$r_3 = r mod 1001$\\

\item [Umgangsprachlich]
$r_1$ wird gesettet auf den Value den man von r mod 999 haven tut.\\
$r_2$ tut similar laufen tun nur mit 1 mehr, namely 1000. Folglich mod 1000.\\
$r_3$ getten wir wieder gleich. But mit 1001-mod.
\end{description}

\end{enumerate}

\end{enumerate}

\end{document}

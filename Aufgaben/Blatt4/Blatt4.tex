%%%%%%%%%%%%%%%%%%%%%%%%%%%%%%%%%%%%%%%%%%%%%%%%%%%%%%%%%%%%%%%%%%%%%%%%%%%%%%%%%%%%%%%%%
%             template.tex                                                              %
%                                                                                       %
%            Author: Sergej Lewin 10/2008                                               %
%                                                                                       %    
% !!!Man braucht noch die Datei Ueb.sty (im gleichen Ordner wie die Hauptdatei)!!!      %
%%%%%%%%%%%%%%%%%%%%%%%%%%%%%%%%%%%%%%%%%%%%%%%%%%%%%%%%%%%%%%%%%%%%%%%%%%%%%%%%%%%%%%%%%
\documentclass[a4paper,11pt]{article}             % bestimmt das Aussehen eines Dokuments
\usepackage{Ueb}                                  % vordefinierte Makros
\usepackage{enumitem}
\renewcommand{\labelenumi}{(\alph{enumi})}
\renewcommand{\labelenumii}{(\roman{enumii})}

%!!!!anpassen an das Betriebssystem!!!, um Umlaute zu verwenden
\usepackage[utf8]{inputenc}                      %Linux
%\usepackage[latin1]{inputenc}                    %Windows
%\usepackage[applemac]{inputenc}                  %Mac



%Namen und Matrikelnummern anpassen
%\zweinamen{Name1}{Matrikelnummer1}{Name2}{Matrikelnummer2} %2er Gruppen
\dreinamen{Alexander Neuwirth}{439218}{Leonhard Segger}{440145}{Jonathan Sigrist}{441760} %3er Gruppe

%Briefkastennummer anpassen. z. B. \briefkasten{104}
\briefkasten{}

%Termin der Uebungsgruppe und Raum anpassen z. B. \termin{Mo. 12-14 , SR2}
\termin{Fr. 08-10, SR217}

%Blattnummer anpassen z. B. \blatt{5}
\blatt{4}

\begin{document}

\Aufgabe{12}

\begin{enumerate}
\item
\begin{enumerate}

\item \texttt{\\
STJ a, a, .+1 \quad a=0
}

\item \texttt{\\
STJ a, a, .+1 \quad a=0\\
STJ t, t, .+1 \quad t=0\\
STJ t, 1, .+1 \quad t=-1\\
STJ a, t, .+1 \quad a=1
}

\item \texttt{\\
STJ a, b, .+1 \quad a=a-b
}

\item \texttt{\\
STJ a, a, .+1 \quad a=0\\
STJ a, b, .+1 \quad a=-b\\
}

\item \texttt{\\
STJ a, a, .+1 \quad a=0\\
STJ t, t, .+1 \quad t=0\\
STJ t, b, .+1 \quad t=-b\\
STJ a, t, .+1 \quad a=b
}

\item \texttt{\\
STJ t, t, .+1 \quad t=0\\
STJ k, k, .+1 \quad k=0\\
STJ t, a, .+1 \quad t=-a\\
STJ k, t, .+1 \quad k=a\\
STJ a, a, .+1 \quad a=0\\
STJ a, k, .+1 \quad a=-a
}

\item \texttt{\\
STJ t, t, .+1 \quad t=0\\
STJ t, b, .+1 \quad t=-b\\
STJ a, t, .+1 \quad a=a+b
}

\item \texttt{\\
STJ 0, 0, x \quad JMP x
}

\item \texttt{\\
STJ t, t, .+1 \quad t=0\\
STJ a, t, x \quad a-0 $\leq$ 0
}

\item \texttt{\\
STJ t, t, .+1 \quad t=0\\
STJ t, a, x \quad 0-a $\leq$ 0
}

\item \texttt{\\
STJ t, t, .+1 \quad t=0\\
STJ a, t, .+2 \quad a $\leq$ 0\\
STJ 0, 0, .+2 \quad else JMP to end\\
STJ t, a, x \quad a $\geq$ 0 
}

\item \texttt{\\
STJ a, b, .+1 \quad a-b $\leq$ 0\\
STJ t, t, .+1 \quad t=0\\
STJ 0, a, x \quad b-a $\leq$ 0 $\Rightarrow$ JMP x
}

\end{enumerate}

\begin{enumerate}
\item \texttt{\\
a:=0\\
if(c<=0) then goto y\\
goto z\\
:label y\\
b:=-b\\
c:=-c\\
:label z\\
if(c>=1) then goto x\\
goto e\\
:label x\\
a:=a+b\\
c:=c-1\\
goto z\\
:label e
}

\item \texttt{\\
if(b<=0) then goto y\\
goto e\\
:label y\\
b:=-b\\
:label e
}

\item \texttt{\\
i:=0\\
ac:=|c|\\
ab:=|b|\\
:label z\\
i:=i+1\\
t:=ac*i\\
if(ab>=t) then goto z\\
i:=i-1\\
t:=b*c\\
if(t<=0) then goto x\\
goto end\\
:label x\\
k:=-1\\
i:=i*k\\
t:=i*c\\
t:=t-b\\
i:=i-1\\
if(t=0) then goto l\\
goto end\\
:label l\\
i:=i+1\\
:label end
}

\item \texttt{\\
if(c=0) then goto k\\
dbc:=b div c\\
t:=c*dbc\\
x:=x-t\\
goto end\\
:label k\\
a:=b\\
:label end
}

\item \texttt{\\
if(b>=c) then goto z\\
a:=b\\
goto end\\
:label z\\
a:=c\\
:label end
}

\item \texttt{\\
:label z\\
if(c=0)then goto end\\
r:=b mod c\\
c:=b\\
b:=r\\
goto z\\
a:=b\\
}

\end{enumerate}

\end{enumerate}

\Aufgabe{13}

\begin{enumerate}
\item
Nach der Matrizenmultiplikation muss gelten:
\begin{equation*}
\begin{array}{c}
A = E \cdot I + F \cdot K\\
B = E \cdot J + F \cdot L\\
C = G \cdot I + H \cdot K\\
D = G \cdot J + H \cdot L
\end{array}
\end{equation*}
Durch die gegebenen Gleichungen ergibt sich
\begin{equation*}
\begin{array}{rcl}
A & = & m_2 + m_3\\
~ & = & E I + F K\\
B & = & t_1 + m_5 + m_6\\
~ & = & m_1 + m_2 + m_5 + m_6\\
~ & = & s_2 s_6 + E I + s_1 s_5 + s_4 L\\
~ & = & (s_1 - E) (L - s_5) + E I + (G + H) (J - I) + (F - s_2) L\\
~ & = & (G + H - E) (L - J + I) + E I + (G + H) (J - I) + (F - G - H + E) L\\
~ & = & GL - GJ + GI + HL - HJ + HI - EL + EJ - EI + EI + GJ - GI\\
~ & ~ & + HJ - HI + FL - GL - HL + EL\\
~ & = & EJ + FL\\
C & = & t_2 - m_7\\
~ & = & t_1 + m_4 - H s_8\\
~ & = & m_1 + m_2 + s_3 s_7 - H (s_6 - K)\\
~ & = & s_2 s_6 + EI + (E - G) (L - J) - H (L - s_5 - K)\\
~ & = & (s_1 - E) (L - s_5) + EI + (E-G) (L-J) - H (L - J + I - K)\\
~ & = & (G+H-E)(L-J+I) + EI + (E-G)(L-J)\\
~ & = & GL-GJ+GI+HL-HJ+HI-EL+EJ-EI+EI\\
~ & ~ & +EL-EJ-GL+GJ-HL+HJ-HI+HK\\
~ & = & GI+HK\\
D & = & t_2 + m_5\\
~ & = & t_1 + m_4 + s_1 s_5\\
~ & = & m_1 + m_2 + s_3 s_7 + (G+H) (J-I)\\
~ & = & s_2 s_6 + EI + (E-G) (L-J) + (G+H) (J-I)\\
~ & = & (s_1-E) (L-s_5) + EI + (E-G) (L-J) + (G+H) (J-I)\\
~ & = & (G+H-E) (L-J+I) + EI + (E-G) (L-J) + (G+H) (J-I)\\
~ & = & GL-GJ+GI+HL-HJ+HI-EL+EJ-EI\\
~ & ~ & +EI+EL-EJ-GL+GJ+GJ-GI+HJ-HI\\
~ & = & HL+GJ
\end{array}
\end{equation*}
Also stimmt der Rechenweg über die Teilrechnungen mit der normalen Matrizenmulltiplikation überein.

\item
Der Algorithmus besteht aus 7 Multiplikationen und 15 Additionen. Nach der Rechnung auf den Folien ergibt sich.
\begin{equation*}
\begin{array}{rl}
T(n) & = 7 T \left ( \frac{n}{2} \right ) + 15 { \left ( \frac{n}{2} \right ) }^2 = 7 \left [ 7 \left ( \frac{n}{4} \right ) + 15 { \left ( \frac{n}{4} \right ) }^2 \right ] + 15 { \left ( \frac{n}{2} \right ) }^2\\
& = \dots \Rightarrow \mathcal{O}(n^{2.807})
\end{array}
\end{equation*}

\end{enumerate}

\Aufgabe{13}

\begin{enumerate}
\item
$(1)$ Sagt aus, dass die Funktion $f$, bis auf einen gewählten festen Faktor $c$, immer (d. h. für alle $n$, da $f$ unabhängig von $n$) kleiner als die Funktion $g$ ist.\\
$(2)$ Sagt aus, dass die Funktion $f$, bis auf einen gewählten festen Faktor $c$, ab einem $n \geq n_0$ immer kleiner als die Funktion $g$ ist.\\
\begin{description}
\item [$(1) \Rightarrow (2)$] offensichtlich; Gilt für alle $n$, mit $n_0 = 0$; $c$ bleibt gleich.
\item [$(2) \Rightarrow (2)$] $c$ muss so gewählt werden, dass $n_0=0$ wird. Dies ist immer möglich, da $c$ beeliebig hoch gewählt werden kann um $f(n) \leq c \cdot g(n)$ zu erfüllen. Wobei immer gilt $g(n) \neq 0$.

\end{description}

\item Damit $\frac{f(n)}{g(n)}$ gegen null konvertiert, müsste $g(n) > f(n)$ ab einem bestimmten $n_0$ gelten und somit $g(n)$ schneller als $f(n)$ wachsen. Da dies eine mächtigere Aussage wie $(2)$ wäre, es also kein $n_0$ gäbe, folgt $f(n) = \mathcal{O} \big (g(n) \big )$.

\item
Die Funktionen $f_7$ und $f_6$ sind komplexer als die Funktionen $f_9$ und $f_{10}$.
\end{enumerate}

\end{document}

